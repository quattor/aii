\section{edg-nbp\label{edg-nbp}\index{edg-nbp}}


Create/remove entries for pxelinux

\subsection*{SYNOPSIS\label{edg-nbp_SYNOPSIS}\index{edg-nbp!SYNOPSIS}}
\begin{verbatim}
 edg-nbp [options] <--boot <hostname>          |
                    --bootlist <filename>      |
                    --configure <hostname> <template> [n1=v1] ... |
                    --configurelist <filename> |
                    --install <hostname>       |
                    --installlist <filename>   |
                    --remove <hostname>        |
                    --removelist <filename>    |
                    --removeall                |
                    --status <hostname>        |
                    --statuslist <filename> >
\end{verbatim}
\subsection*{DESCRIPTION\label{edg-nbp_DESCRIPTION}\index{edg-nbp!DESCRIPTION}}


edg-nbp manage configuration the Network Boot Program (NBP) pxelinux.
It creates the required configuration/symlink for an host to prepare it for
the installation or for normal; it can also remove previously added hosts.



It is normally executed by edg-shellfe but it can be run as stand-alone. Input data
can be specified via command line (for just 1 host) or via text file (for more hosts).



Command line line options override default values in \$EDG\_LOCATION/etc/edg-nbp.conf.

\subsection*{COMMANDS\label{edg-nbp_COMMANDS}\index{edg-nbp!COMMANDS}}
\begin{description}

\item[--boot $<$hostname$>$] \mbox{}

Enable for $<$hostname$>$ the boot from local disk (i.e do not reinstall
the OS).


\item[--bootlist $<$filename$>$] \mbox{}

Enable for hosts listed on $<$filename$>$ normal boot from local disk.


\item[--configure $<$hostname$>$ $<$template$>$] \textbf{[n1=v1] [n2=v2] ...}

Configure $<$hostname$>$ for NBP using $<$template$>$ as basic
configuration files. If $<$template$>$ is just a filename it will be
read from the directory specied via --nbpdir (see below,
default value: \$EDG\_LOCATION/lib/aii/nbp). The parameters to write
are expressed as couples name=value, where name is an id that
has to be present in the template (see the example at the end).



If the node is already present its configuration is simply
replaced by the new one.


\item[--configurelist $<$filename$>$] \mbox{}

Configure hosts listed on $<$filename$>$. Hosts have to be listed one per line
with the simple syntax $<$hostname$>$ $<$template$>$ [n1="v1"] [n2="v2"], where $<$hostname$>$
and $<$template$>$ are mandatory. Lines with \# are comment.
Values must be expressed between double (") quotes.
An example:

\begin{verbatim}
 # This is an unuseful comment
 node2.qwer.fi rh73-template  label="Redhat 7.3" kernel="vmlinuz-2.4.18"
 node2.qwer.fi rh90-template  label="Redhat 9.0" kernel="vmlinuz-2.4.20"
\end{verbatim}

\item[--install $<$hostname$>$] \mbox{}

Enable OS installation via network for $<$hostname$>$.


\item[--installlist $<$filename$>$] \mbox{}

Enable OS installation via network for hosts listed on $<$filename$>$.
Hosts have to be listed one per line. Lines starting with \# are comment.


\item[--remove $<$hostname$>$] \mbox{}

Remove configuration file for $<$hostname$>$.


\item[--removelist $<$filename$>$] \mbox{}

Remove configurations for hosts listed on $<$filename$>$. Hosts have to
be listed one per line. Lines starting with \# are comment.


\item[--removeall] \mbox{}

Remove configurations for *ALL* hosts configured. Useful only in case
of problems/test.


\item[--status $<$hostname$>$] \mbox{}

Report the boot status (boot from local disk/install) for $<$hostname$>$.


\item[--statuslist $<$filename$>$] \mbox{}

Report the boot status (boot from local disk/install) for hosts listed
on $<$filename$>$. Hosts have to be listed one per line. Lines starting
with \# are comment.

\end{description}
\subsection*{OPTIONS\label{edg-nbp_OPTIONS}\index{edg-nbp!OPTIONS}}
\begin{description}

\item[--bootconfig $<$filename$>$] \mbox{}

Generic "boot from local disk" NBP configuration file (default:
localboot.cfg). It must be present in 'templatedir' (see below).
It is "symlinked" by edg-nbp script into 'nbpdir' (see below).


\item[--cfgfile $<$path$>$] \mbox{}

Use as configuration file $<$path$>$ instead of the default
configuration file \$EDG\_LOCATION/etc/edg-nbp.conf.


\item[--nbpdir $<$directory$>$] \mbox{}

Directory where the NBP configuration files have to be stored
(default: /osinstall/nbp/pxelinux.cfg). If it does not exist, it will
be automatically created.


\item[--templatedir $<$directory$>$] \mbox{}

Directory where there the template files are present
(default: /opt/edg/lib/aii/nbp).

\end{description}
\subsection*{CONFIGURATION FILE\label{edg-nbp_CONFIGURATION_FILE}\index{edg-nbp!CONFIGURATION FILE}}


Default values of command lines options can be specified in the file
\$EDG\_LOCATION/etc/edg-nbp.conf using syntax:

\begin{verbatim}
 <option> = <value>
\end{verbatim}


e.g.:

\begin{verbatim}
 nbpdir = /my/personal/nbp
\end{verbatim}
\subsection*{AUTHORS\label{edg-nbp_AUTHORS}\index{edg-nbp!AUTHORS}}


Enrico Ferro $<$enrico.ferro@pd.infn.it$>$

\subsection*{MORE INFORMATION\label{edg-nbp_MORE_INFORMATION}\index{edg-nbp!MORE INFORMATION}}


Pxelinux: http://syslinux.zytor.com.

